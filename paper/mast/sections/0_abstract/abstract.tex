\documentclass[a4paper,11pt]{ujreport}
%%【PostScript, JPEG, PNG等の画像の貼り込み】
%% 利用するパッケージを選んでコメントアウトしてください.
\usepackage{graphicx} % for \includegraphics[width=3cm]{sample.eps}
\usepackage{epsfig} % for \psfig{file=sample.eps,width=3cm}
%\usepackage{epsf} % for \epsfile{file=sample.eps,scale=0.6}
%\usepackage{epsbox} % for \epsfile{file=sample.eps,scale=0.6}
\usepackage{src/class/mediabb} % for pdf

\usepackage{times} % use Times Font instead of Computer Modern
% \usepackage{listings} % for soursecode
% \usepackage{plistings} % for soursecode
\usepackage{src/class/docmute} % texファイル分割用

\setcounter{tocdepth}{3}
\setcounter{page}{-1}

\setlength{\oddsidemargin}{0.1in}
\setlength{\evensidemargin}{0.1in}
\setlength{\topmargin}{0in}
\setlength{\textwidth}{6in}
%\setlength{\textheight}{10.1in}
\setlength{\parskip}{0em}
\setlength{\topsep}{0em}

%\newcommand{\zu}[1]{{\gt \bf 図\ref{#1}}}

%% タイトル生成用パッケージ(重要)
\usepackage{src/class/mast-jp-sjis}

%% タイトル
%% 【注意】タイトルの最後に\\ を入れるとエラーになります
\title{NoSQL型データベースシステムでの実体化ビュー選択に関する研究}
%% 著者
\author{髙木 颯汰}
%% 指導教員
\advisor{古瀬 一隆 陳 漢雄}

%% 年月 (提出年月)
%% 年月は必要に応じて書き替えてください.
\majorfield{ } \yearandmonth{2019年 1月}


\addtocounter{page}{2} %単体でコンパイルした際の調整用
\begin{document}

\begin{center}
	{\bf 概要}
	\vspace{5mm}
\end{center}
本論文ではNoSQLの一種であるドキュメント指向データベースに実体化ビューを導入する事によって問い合わせ処理を高速化する手法を提案する.ドキュメント指向データベースでは従来のリレーショナルデータベースにあったような参照型のデータ構造に加えて埋込型のデータ構造を選択できる.参照先の内容を埋め込む事によって結合処理をしなくて済むが,ファイルサイズが大きくなる傾向にあり,フラグメンテーションが発生し,逆にパフォーマンスが落ちる可能性がある.そこで本手法ではリレーショナルデータベースで実現されている実体化ビューの概念をNoSQLにも応用する事で,問い合わせ処理を自動的に高速化する.具体的には,頻繁に問い合わせのある結合処理や集計処理を自動的に検知してその部分のみ予め実体化することでデータベースアクセスの高速化を実現している.実体化する箇所の選択を自動化することにより,データベースシステム管理者が行なっていた作業を簡略化し,客観的で正確な実体化ビュー選択が可能となる.

\end{document}
