\documentclass[a4paper,11pt]{ujreport}
%%【PostScript, JPEG, PNG等の画像の貼り込み】
%% 利用するパッケージを選んでコメントアウトしてください.
\usepackage{graphicx} % for \includegraphics[width=3cm]{sample.eps}
\usepackage{epsfig} % for \psfig{file=sample.eps,width=3cm}
%\usepackage{epsf} % for \epsfile{file=sample.eps,scale=0.6}
%\usepackage{epsbox} % for \epsfile{file=sample.eps,scale=0.6}
\usepackage{src/class/mediabb} % for pdf

\usepackage{times} % use Times Font instead of Computer Modern
% \usepackage{listings} % for soursecode
% \usepackage{plistings} % for soursecode
\usepackage{src/class/docmute} % texファイル分割用

\setcounter{tocdepth}{3}
\setcounter{page}{-1}

\setlength{\oddsidemargin}{0.1in}
\setlength{\evensidemargin}{0.1in}
\setlength{\topmargin}{0in}
\setlength{\textwidth}{6in}
%\setlength{\textheight}{10.1in}
\setlength{\parskip}{0em}
\setlength{\topsep}{0em}

%\newcommand{\zu}[1]{{\gt \bf 図\ref{#1}}}

%% タイトル生成用パッケージ(重要)
\usepackage{src/class/mast-jp-sjis}

%% タイトル
%% 【注意】タイトルの最後に\\ を入れるとエラーになります
\title{NoSQL型データベースシステムでの実体化ビュー選択に関する研究}
%% 著者
\author{髙木 颯汰}
%% 指導教員
\advisor{古瀬 一隆 陳 漢雄}

%% 年月 (提出年月)
%% 年月は必要に応じて書き替えてください.
\majorfield{ } \yearandmonth{2019年 1月}


\addtocounter{page}{2} %単体でコンパイルした際の調整用
\begin{document}

\chapter{まとめ}
\label{chap:Conclusion}
リレーショナルデータベースシステムで用いられる技術の一つの実体化ビューは負荷が大きい処理を事前に実データとして保持しておくことで,システムを高速化することができる.しかし,実体化ビューを作成する箇所の選択やメンテナンスを適切に行わないと,返ってボトルネックになってしまう可能性がある.本研究ではNoSQLの一種であるドキュメント指向型データベースにおける実体化ビュー選択の自動化を提案する.
提案手法ではミドルウェアを実装し,集積したユーザーからのクエリのログを実体化する箇所の判定に用いた.また,クエリのログや処理時間などの情報を元にボトルネックとなっている実体化ビューを探して,元のデータモデルに戻す実装を行った.

実験では,提案手法と全体を実体化したコレクションと元のデータモデルを比較し,提案手法の有用性を確かめた.実験の結果,提案手法と全体を実体化したコレクションが元のデータモデルと比べて検索処理時間が高速であった.実体化ビューを用いると更新処理速度が低下する傾向にあったが,あるクエリ条件下では提案手法が検索処理速度を保ちつつ,全体を実体化したコレクションの約2倍の更新速度を実現した.

今後は,課題となった更新速度を克服するために実体化条件と逆実体化条件に処理時間削減率等の概念を追加し,さらなる機能向上を目指す.また,複雑なクエリパターンや時系列でクエリパターンが変化する状況でも提案手法の有用性を示せるように,ミドルウェアの性能を向上させる必要がある.

\end{document}
